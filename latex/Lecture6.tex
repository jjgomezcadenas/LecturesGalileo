\documentclass [aspectratio=169]{beamer}
\beamertemplatenavigationsymbolsempty
\usetheme{Boadilla}
\usepackage{textpos} % package for the positioning
\usepackage[]{graphicx}
\usepackage{graphicx}
\usepackage{float}
\usepackage{hyperref}
\usepackage{caption}
\usepackage{subcaption}
\usepackage{algorithm,algpseudocode}
\usepackage[export]{adjustbox}
\usepackage{tikz}
\usepackage[square,numbers]{natbib}
\usepackage[byname]{smartref}
\usetikzlibrary{positioning}
\usetikzlibrary{arrows, shapes, decorations, automata, backgrounds, fit, petri, calc}

\newcommand*{\logofont}{\fontfamily{phv}\selectfont}

\definecolor{uwopurple}{RGB}{79,38,131} % official purple color for uwo

\title[]{\vspace{60pt} \\
Liquid Times} % Change the lecture topic right here!
\subtitle{The EXO/nEXO way}
\author[]{J.J. Gómez Cadenas}
\institute[]{Donostia International Physics Center}
\date{\today}

% Math notations
\newtheorem{thm}{Theorem}[section]
\newtheorem{lem}[thm]{Lemma}

\newtheorem{defn}[thm]{Definition}
\newtheorem{eg}[thm]{Example}
\newtheorem{ex}[thm]{Exercise}
\newtheorem{conj}[thm]{Conjecture}
\newtheorem{cor}[thm]{Corollary}
\newtheorem{claim}[thm]{Claim}
\newtheorem{rmk}[thm]{Remark}

\newcommand{\ie}{\emph{i.e.} }
\newcommand{\cf}{\emph{cf.} }
\newcommand{\into}{\hookrightarrow}
\newcommand{\dirac}{\slashed{\partial}}
\newcommand{\bbonu}{\ensuremath{\beta\beta0\nu}}
\newcommand{\bbtnu}{\ensuremath{\beta\beta2\nu}}
\newcommand{\mbb}{\ensuremath{m_{\beta\beta}}}
\newcommand{\qbb}{\ensuremath{Q_{\beta\beta}}}
\newcommand{\mbbsq}{\ensuremath{m_{\beta\beta}^2}}
\newcommand{\tonu}{\ensuremath{(T_{1/2}^{0\nu})^{-1}}}
\newcommand{\gonu}{\ensuremath{G^{0\nu}}}
\newcommand{\monu}{\ensuremath{| M^{0\nu}|^2}}
\newcommand{\XE}{\ensuremath{{}^{136}{\rm Xe}}}
\newcommand{\XES}{\ensuremath{{}^{137}{\rm Xe}}}
\newcommand{\GE}{\ensuremath{{}^{76}{\rm Ge}}}
\newcommand{\TE}{\ensuremath{{}^{130}{\rm Te}}}
\newcommand{\TL}{\ensuremath{{}^{208}{\rm Tl}}}
\newcommand{\BI}{\ensuremath{{}^{214}{\rm Bi}}}
\newcommand{\MO}{\ensuremath{{}^{100}{\rm Mo}}}
\newcommand{\KR}{\ensuremath{{}^{83}{\rm Kr}}}
\newcommand{\nne}{\ensuremath{\bar{N}_e}}
\newcommand{\nng}{\ensuremath{\bar{N}_\gamma}}
\newcommand{\so}{\ensuremath{\rm S_1}}
\newcommand{\st}{\ensuremath{\rm S_2}}
\newcommand{\tz}{\ensuremath{\rm t_0}}
\newcommand{\R}{\mathbb{R}}
\newcommand{\C}{\mathbb{C}}
\newcommand{\Z}{\mathbb{Z}}
\newcommand{\N}{\mathbb{N}}
\newcommand{\Q}{\mathbb{Q}}
\newcommand{\LieT}{\mathfrak{t}}
\newcommand{\T}{\mathbb{T}}
\newcommand{\A}{\mathds{A}}
\newcommand{\E}{\mathbb{E}}
\newcommand{\Prob}{\mathbb{P}}
\newcommand{\Var}{\text{Var}}
\newcommand\equalhat{%
\let\savearraystretch\arraystretch
\renewcommand\arraystretch{0.3}
\begin{array}{c}
\stretchto{
    \scalerel*[\widthof{=}]{\wedge}
    {\rule{1ex}{3ex}}%
}{0.5ex}\\ 
=%
\end{array}
\let\arraystretch\savearraystretch
}

% set color
\setbeamercolor{title in head/foot}{bg=white}
\setbeamercolor{author in head/foot}{bg=white}
\setbeamercolor{date in head/foot}{fg=uwopurple}
\setbeamercolor{date in head/foot}{bg=white}
\setbeamercolor{title}{fg=uwopurple}
\setbeamerfont{title}{series=\bfseries}
\setbeamercolor{frametitle}{fg=uwopurple}
\setbeamerfont{frametitle}{series=\bfseries}
\setbeamercolor{block title}{bg=uwopurple!30,fg=black}
\setbeamercolor{item}{fg=uwopurple}
\setbeamercolor{caption name}{fg=uwopurple!70!}


% set logo at non-title pages
%\logo{\includegraphics[height=0.9cm]{dipc.png}\vspace*{-.45\paperheight}\hspace*{.50\paperwidth}}

\begin{document}

{
\setbeamertemplate{logo}{}
\begin{frame}
    \titlepage
    \begin{textblock*}{4cm}(0.5cm,-7.3cm)
        \includegraphics[width=4cm]{dipc.png}
    \end{textblock*}
    \begin{textblock*}{8cm}(5.0cm,-7.0cm)
        \huge \color{uwopurple}{$\Bigr\rvert$ \hspace{0.15cm} \textbf{Lecture 6}} % Change the lecture # right here! 
    \end{textblock*}
\end{frame}
}

%%%%
\begin{frame}{The LXe (single phase) TPC}
\begin{columns}
\column{0.50\textwidth}
\includegraphics[scale=0.26]{dualLXe.png}

$\bullet~$ {\bf Two-phase LXe-TPC}: same principle that the HPXe TPC (NEXT). Reads primary scintillation (\so) and a delayed secondary scintillation (\st) produced by EL amplification (in gas). 

\column{0.50\textwidth}
 \begin{textblock*}{8cm}(0.5cm, -3.4cm)
 \includegraphics[scale=0.20]{nexobasis.png}
       % \includegraphics[width=4cm]{dipc.png}
       
       $\bullet~$ {\bf Single-phase LXe-TPC}: Reads primary scintillation (\so) and collects the ionisation charge in wires (no EL amplification). 

    \end{textblock*}


%$\bullet~$ {\bf Single-phase LXe-TPC}: Reads primary scintillation (\so) and collects the ionisation charge in wires (no EL amplification). 


%64,000 e/MeV 

%Q = E/W, W=15.6 eV. (13.8 at zero drift field). 

\end{columns}
\end{frame}

\begin{frame}{The LXe (single phase) TPC}
\begin{columns}
\column{0.50\textwidth}
\includegraphics[scale=0.6]{nexosketch.png}

\column{0.50\textwidth}
$\bullet~$ {\bf Single-phase advantages}: simplicity, resulting in fewer components and lower background. 

$\bullet~$ {\bf Single-phase disadvantages}: no amplification of \st, thus higher threshold. Thus, \KR\ calibration no possible (but there are other possibilities). At 2.5 MeV, $Q = 2.5\times 10^6/15.6 = 160\, 256$~electrons, enough to measure the charge with good resolution.  
\end{columns}
\end{frame}

\begin{frame}{The LXe (asymmetric) TPC}
\begin{columns}
\column{0.50\textwidth}
\includegraphics[scale=0.6]{nexosketch.png}

\column{0.50\textwidth}
$\bullet~$ {\bf asymmetric advantage}: Grids (thus radon decays) in the edges of the fiducial volume (unlike the case of NEXT, where the gas is rather transparent, this is a must, otherwise shielding inefficient). 

$\bullet~$ {\bf asymmetric disadvantages}: Double drift distance and potencial at the cathode (significant amount of LXe lost in buffer).  
\end{columns}
\end{frame}

\begin{frame}{The LXe (instrumented barrel) TPC}
\begin{columns}
\column{0.50\textwidth}
\includegraphics[scale=0.6]{nexosketch.png}

\column{0.50\textwidth}
$\bullet~$ {\bf advantage}: Avoid PMTs which are rather radioactive (same argument for NEXT). 

$\bullet~$ {\bf disadvantage}: Lots of SiPMs!. 

$\bullet~$ {\bf Exercise}: Why not a BFD like NEXT-HD?  
\end{columns}
\end{frame}
%%%%

%\begin{frame}{Gamma attenuation in LXe}
%\begin{columns}
%\column{0.50\textwidth}
%\includegraphics[scale=0.2]{attLxe.png}
%
%$\bullet~$ Consider a LXe TPC with $L = 125$~ cm and $D = 116$~ cm (e.g, similar to nEXO). It will host 4.1 tons of LXe.
%
%$\bullet~$ Define a fiducial region as a volume, centred in the TPC of 
%$l = 86$~ cm and $d = 71$~ cm. It will host $\sim$1 ton of LXe
%
%\column{0.50\textwidth}
%$\bullet~$ The shielding length for gammas emitted from the end-caps is 
%$Z = (L - l)/2 = 19.5$~cm, and $R = (D-d)/2 = 22$~cm.  
%
%$\bullet~$ The fraction of the gamma flux emitted from the envelop interacting along the longitudinal axis is:  $ e^{-Z/\lambda_{att}}$
%
%$\bullet~$  In LXe: $\lambda_{att} = 8.7~$ cm. Then $e^{-19.5/8.5} = 0.1$.
% 
% $\bullet~$ Take away: shielding provides one order of magnitude rejection to the worst type of gammas (those coming from \BI) with an efficiency of 20\%. 
%\end{columns}
%\end{frame}

\begin{frame}{Self-shielding in LXe}
%\begin{columns}
%\column{0.60\textwidth}
\includegraphics[scale=0.35]{self-shielding-lxe.png}
\end{frame}


\begin{frame}{Energy resolution in LXe}
%\begin{columns}
%\column{0.60\textwidth}
\includegraphics[scale=0.35]{scintioniexo.png}


%\column{0.40\textwidth}
$\bullet~$ Using anti-correlation between scintillation and ionisation yields a good combined energy resolution. EXO measured $\sigma = 1.15$\% at 2.6 MeV. This translates into  
$\sigma_{FWHM} = 3.5$\% at \qbb.  
$\bullet~$ Energy resolution can be improved with better light collection: nEXO expects 2.2\% at \qbb. 

%\end{columns}
\end{frame}

%%%
\begin{frame}{Topology in LXe}
\begin{columns}
\column{0.50\textwidth}
\includegraphics[scale=0.35]{exoTopo.png}


\column{0.50\textwidth}
$\bullet~$ A LXe TPC has the capability to separate single-site from multi-site events.  

$\bullet~$ This is extremely useful to reject pile-up events from low-energy gammas, as well as Compton interactions of the  \BI\ and \TL\ $\gamma$'s when the second blob can be identified.   

\end{columns}
\end{frame}
%%%

\begin{frame}{LXe handles to suppress backgrounds}
%\begin{columns}
%\column{0.60\textwidth}
\includegraphics[scale=0.35]{nexto-tools.png}

$\bullet~$ Energy resolution is good enough to suppress \bbtnu\ as well as to reduce many other backgrounds (including, notably the one due to \TL\ or the internal background due to \XES). Not 
enough to help in the case of \BI\ (topology is not useful either to suppress the photoelectric interactions of the \BI\ gamma of 2447 keV). This represents the ``worst case'' background for LXe (and also for GXe) and we will study it in our Research Project. 
\end{frame}

\begin{frame}{EXO}
\includegraphics[scale=0.40]{exo.png}
\end{frame}

\begin{frame}{EXO TPC}
\includegraphics[scale=0.40]{exo-tpc.png}
\end{frame}

\begin{frame}{EXO TPC}
\includegraphics[scale=0.40]{apds.png}
\end{frame}

\begin{frame}{EXO Spectrum}
\begin{columns}
\column{0.70\textwidth}
\includegraphics[scale=0.40]{exo-spectrum.tex}
\column{0.30\textwidth}
$\bullet~$ Notice that the MS (multiple side) data can be used to characterise backgrounds.  

$\bullet~$ SS data corresponds to signal (plus backgrounds that mimic it).   

\end{columns}
\end{frame}

%nexo-sensors.png
%0.7 % sigma at 2614
\begin{frame}{EXO: Search for \bbonu}
\begin{columns}
\column{0.70\textwidth}
\includegraphics[scale=0.35]{exo-fit.png}
\column{0.30\textwidth}
$\bullet~$ Assuming 100 kg (fiducial mass) and a resolution of 3.5\% this implies about 6 events in
1 $\sigma~$ ROI.   

$\bullet~$ nEXO's goal is to reduce this background by one order of magnitude while increasing the fiducial mass also by an order of magnitude (the total mass is increased by 25).    

\end{columns}
\end{frame}


\begin{frame}{From EXO to nEXO}
\includegraphics[scale=0.30]{exonexo.png}
\end{frame}

%%%

\begin{frame}{Double shell cryostat}
\includegraphics[scale=0.30]{nexoinsnow.png}
\end{frame}

%%%
\begin{frame}{nEXO would operate deep underground (SNOWLAB)}
\includegraphics[scale=0.30]{nexosnowlab.png}
\end{frame}

%%%

\begin{frame}{Key elements of nEXO}
\begin{columns}
\column{0.50\textwidth}
\includegraphics[scale=0.20]{nexokey.png}
\column{0.50\textwidth}
$\bullet~$ Key elements are very similar to those of NEXT-HD: The TPC vessel is made of ultra-pure copper, and the active elements are the charge-tiles (readout of ionisation) and photon detector system (readout of scintillation). The challenge in nEXO, like in NEXT is to achieve a level of radiopurity as high as possible in those components. 
\end{columns}
\end{frame}

%%%
\begin{frame}{Light readout}
\includegraphics[scale=0.25]{nexoLightReadout.png}

$\bullet~$: In NEXT-HD, a ``dense tracking plane" could use similar ideas. The main advantage of nEXO is that the detector is cold, thus SiPMs DC is very small.  
\end{frame}

%%%

%%%
\begin{frame}{Charge detection system}
\includegraphics[scale=0.25]{nexocharge.png}
\end{frame}

%%%

\begin{frame}{Charge detection system}
\includegraphics[scale=0.25]{chargeReadoutnexo.png}

$\bullet~$ Tradeoff: more material (electronics) near the active volume. 
\end{frame}

\begin{frame}{Electroformed copper vessel}
\includegraphics[scale=0.30]{electroformedcopper.png}

$\bullet~$ Useful technique for NEXT-HD too! ICS last radiation length made of same stuff than NEXO.  
\end{frame}

\begin{frame}{Field Cage}
\includegraphics[scale=0.30]{nexoFieldCage.png}

$\bullet~$ Lots of common ground here between NEXT and nEXO (copper, resistors, grids, HV, etc.).  
\end{frame}

\begin{frame}{NEXT and nEXO handles}
\includegraphics[scale=0.30]{nexoHandles.png}
\includegraphics[scale=0.30]{nextSignatures.png}

$\bullet~$ Both detectors use energy resolution and topology as handles. 
 
 $\bullet~$nEXO golden handle: self-shielding.
 
 $\bullet~$NEXT  golden handle: Excellent energy resolution and two-electron identification. 
   
\end{frame}



%\begin{frame}{Gamma transparency in GXe}
%\begin{columns}
%\column{0.50\textwidth}
%\includegraphics[scale=0.2]{nexttrans.png}
%
%$\bullet~$ Consider a cylinder of $L=D=1~ m^3$
%
%$\bullet~$ Condition of transparency is that $\gamma$ crosses the full volume
%\column{0.50\textwidth}
% $\mu/\rho = 4.8 \times 10^{-2}$~cm$^2$/g, gas STP, $\rho = 5.37$~g/m$^3$.
% 
% $\mu = 5.37 \times 4.8 \times 10^{-6} = 2.6 \times 10^{-5}$~m$^{-1}$.
% 
% Interacting: $ 1- e^{-\mu L} \sim 1 - e^{-2.6 \times 10^{-5}} = 2.6  \times 10^{-5}$
% 
% At 10 bar, $I \sim 2.6 \times 10^{-4}$
% 
% 
%\end{columns}
%\end{frame}



%%%
\end{document}